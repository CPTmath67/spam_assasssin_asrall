\documentclass[a4paper,11pt]{article}
\usepackage[utf8]{inputenc}
\usepackage[francais]{babel}
\usepackage{graphicx}
\usepackage{fancyhdr}
\usepackage {color}
\usepackage{colortbl}

% Title Page
\title{SPAM ASSASSIN}
\date{Décembre 2014}
\author{Mathieu KERN - Aymeric HINDERCHIETTE}
\pagestyle{fancy}


\begin{document}

\maketitle
\includegraphics{spamassassinintro.png}
\pagebreak

\tableofcontents

\pagebreak

\part{Présentation de SpamAssassin}

\section{Problématique }

  Le mail (ou courriel) est aujourd'hui le moyen privilégié de communication à travers le monde. Massivement utilisé, 
d'une certaine fiabilité et éprouvé par des décennies d'utilisation il reste le moyen le plus répandu pour 
les communications entres les personnes. Malheureusement, mail est également aujourd'hui synonyme de spam, ces messages
indésirables qui s'entassent dans nos boites mails. C'est ici qu'entre en jeu SpamAssassin.

\subsection{Le SPAM}
Avant de poursuivre sur SpamAssassin, rappelons concrètement ce qu'est le SPAM et ce qu'il implique. 

\paragraph{Comment reconnaître un SPAM :}

\begin{itemize}
 \item De par sa nature, un SPAM n'est pas désiré par l'utilisateur qui le reçoit. 
 \item La réception d'un SPAM résulte d'un envoi massif : une machine (souvent un bot) envoie le même message 
 à plusieurs destinataires sans aucun discernement. Cela s'oppose aux messages ciblés par exemple de commerçant,
 qui n’envoie que à leurs prospects (à but publicitaire).
 \item Son contenu n'est pas spécifiquement destiné à l'utilisateur (chaque personne reçoit le même contenu).
 \item Une importante liste de destinataires. 
 \item Entête des messages souvent corrompues ou ne respectant pas les normes.
\end{itemize}

\paragraph{Statut légal}


La loi pour la confiance dans l'économie numérique du 21 juin 2004 contient une transposition de la
directive européenne du 12 juillet 2002\footnote{Le principe introduit figura également à l'article L.34.5 du code 
des postes et des communications électroniques } relative à la protection de la vie privée dans le secteur des communications
électroniques:
\begin{quote}
 Est interdite la prospection directe au moyen d'un automate d'appel, d'un télécopieur ou d'un courrier électronique utilisant,
 sous quelque forme que ce soit, les coordonnées d'une personne physique qui n'a pas exprimé son consentement préalable à recevoir
 des prospections directes par ce moyen. 
\end{quote}
Les SPAM sont donc connus du droit français et encadrés par des textes spécifiques.

\section{Le projet}

\subsection{Informations}


\begin{center}
\begin{tabular}{cll}
\hline
Développeur & Apache Software Foundation  \\
Langage & Perl 
Dernière version & 3.4.0 (11 février 2014) [+/-] \\
Environnements & Multiplate-forme  \\
Type & Anti-spam & \\
Licence & Licence Apache 2.0 \\ \\
\hline
\end{tabular}
\end{center}



SpamAsassin est donc aujourd'hui sous le giron de la Apache Software Foundation, organisation à but non lucratif qui
s'occupe également du serveur Apache, Logiciel de distribution de contenu WEB le plus utilisés au monde. 
Elle gère également 150 autres projets.
EN outre tout ses projets sont distribués sous sa propre Licence, la licence Apache(actuellement en 2.0) , qui est compatible GPL v3.
Cette licence met l’accent sur le copyright tout en restant bien sur une licence libre. Les objectifs principaux de la Fondation sont de protéger 
juridiquement le travail des contributeurs et d'empêcher que la marque Apache soit utilisée illégalement.

Le projet SpamAssassin est actif depuis plus d'une décennie et est constamment en développement 
pour s'adapter aux développements des méthodes qu'utilisent les spammeurs. C'est en outre le programme anti-spam le plus utilisé à cause de son efficacité.

\subsection{Développement}

SpamAssasin contient environ 300 000 lignes de codes ce qui en fait un très gros projet( Graphique ~\ref{fig:code}).
Le projet est à maturité et il ne grossit plus depuis plusieurs années, les développeurs se concentrant sur l'optimisation du code existant.
Il y actuellement 23 développeurs, avec une répartition des lignes codes assez inégales, notamment deux développeurs qui ont fait la majorité du code( Tableau ~\ref{tab:devs})

\begin{figure}
 \includegraphics[width=\textwidth]{annexes/lignes.png}
  \caption{Évolution du nombre de ligne de codes}
  \label {fig:code}
\end{figure}

\begin{table}
 
\definecolor{tcA}{rgb}{0.627451,0.627451,0.643137}
\begin{center}
\begin{tabular}{lllll}

\rowcolor{tcA}
 Author Id & Changes & Lines of Code & Lines per Change\\
\rowcolor{tcA}
 Totals & 26092 (100.0\%) & 1403447 (100.0\%) & 53.7\\
\rowcolor{tcA}
  jm & 8136 (31.2\%) & 721593 (51.4\%) & 88.6\\
\rowcolor{tcA}
  spamassassin role & 7997 (30.6\%) & 463448 (33.0\%) & 57.9\\
\rowcolor{tcA}
  axb & 741 (2.8\%) & 54525 (3.9\%) & 73.5\\
\rowcolor{tcA}
  mmartinec & 1779 (6.8\%) & 32348 (2.3\%) & 18.1\\
\rowcolor{tcA}
  felicity & 1625 (6.2\%) & 29134 (2.1\%) & 17.9\\
\rowcolor{tcA}
  kmcgrail & 605 (2.3\%) & 21294 (1.5\%) & 35.1\\
\rowcolor{tcA}
  quinlan & 1100 (4.2\%) & 19583 (1.4\%) & 17.8\\
\rowcolor{tcA}
  parker & 309 (1.2\%) & 10407 (0.7\%) & 33.6\\
\rowcolor{tcA}
  khopesh & 1369 (5.2\%) & 10296 (0.7\%) & 7.5\\
\rowcolor{tcA}
  dos & 427 (1.6\%) & 8427 (0.6\%) & 19.7\\
\rowcolor{tcA}
  jhardin & 1021 (3.9) & 6845 (0.5\%) & 6.7\\
\rowcolor{tcA}
  wtogami & 137 (0.5\%) & 6470 (0.5\%) & 47.2\\
\rowcolor{tcA}
  hstern & 63 (0.2\%) & 6029 (0.4\%) & 95.6\\
\rowcolor{tcA}
  sidney & 275 (1.1\%) & 3552 (0.3\%) & 12.9\\
\rowcolor{tcA}
  jquinn & 28 (0.1\%) & 2181 (0.2\%) & 77.8\\
\rowcolor{tcA}
  mss & 133 (0.5\%) & 2072 (0.1\%) & 15.5\\
\rowcolor{tcA}
 hege & 95 (0.4\%) & 1931 (0.1\%) & 20.3\\
\rowcolor{tcA}
 duncf & 109 (0.4\%) & 1892 (0.1\%) & 17.3\\
\rowcolor{tcA}
  jgmyers & 66 (0.3\%) & 508 (0.0\%) & 7.6\\
\rowcolor{tcA}
  smf & 35 (0.1\%) & 499 (0.0\%) & 14.2\\
\rowcolor{tcA}
  maddoc & 11 (0.0\%) & 319 (0.0\%) & 29.0\\
\rowcolor{tcA}
  fanf & 12 (0.0\%) & 49 (0.0\%) & 4.0\\
\rowcolor{tcA}
  kb & 19 (0.1\%) & 45 (0.0\%) & 2.3
\end{tabular}
\caption{Statistique des développeurs du projets }
\footnotesize{Source \footnote{Tableau généré avec StatSVN à partir des sources de SpamAssassin}}
\label{tab:devs}
\end{center}
\end{table}

\subsection{Qu'est ce que SpamAssassin}

SpamAssassin est un programme écrit en PERL dont le but est de filtrer activement les Emails en se basant sur des mécanismes internes. 
SpamAssassin n'effectue aucune action envers les mails, il ajoute seulement des informations personnalisés
qui peuvent être utilisée par d'autres programmes pour effectuer des actions sur les mails (les ranger des dossiers distincts, les supprimer, les bloquer, \dots)


Il peut être utilisé de plusieurs manières :
\begin{itemize}
 \item En mode client, lancé à chaque fois que l'on fait appel à lui
 \item En mode demon grâce à \emph{spamd} , les appels au demon étant fait avec l'utilitaire \emph{spamc}.
 \item Comme une interface de programation: des programmes qui nécessitent des fonctionnalités de filtrage de SPAM peuvent s'interfacer avec SpamAssin
 pour construire des solutions utilisant ses fonctionnalités
\end{itemize}

\pagebreak

\section{Ses fonctionnalités}

\subsection{Comment il filtre}

SpamAssassin reçoit des mails que lui redirigent d'autres programmes, y effectue des tests pour déterminer si ce sont des SPAMS, 
puis renvoie les mails testés au programme qui les as envoyés.

\paragraph{Les tests de SpamAssassin }
\begin{description}
 \item [Champs d'entête] En se basant sur la forme des entêtes et en les comparant avec des schéma connus par SpamAssasin. En effet 
 on peut se base sur la façon dont certains systèmes de SPAMs construisent leurs messages pour les filtrer. 
 \item [Corps du message] Bien sur SpamAssasin permet de filtrer les mails suivant les mots et expressions qu'ils contiennent. 
 ``ceci n'est pas un SPAM'', ``Bonjour jes suis une princesse d'un royaume africain'', ``Venez chechez votre lot'', \dots sont des expressions
 typiques pour des SPAMs. 
 \item [Filtre bayésien] Filtrer les entêtes et le corps d'un message résultera toujours en de multiples faux positifs. C'est ici que le filtrage 
 bayésien se révèle interessant car il va prendre en considération ce que l'on considère comme SPam et non SPAM soit des ``bon mails'' (``HAM'' en anglais).
 Il va ensuite utiliser les repertoires de SPAMs connu et de ``HAM``  connus, pour y identifier les mots et phrases (Définits comme ''Tokens`` en anglais)
 qui n'apparaissent que dans les SPAMs et que dans les ''HAMs``.
 Un token SPAM trouvé resultant d'une hausse du score (voir \ref{score}) SPAM, un token résultant en une baisse de ce niveau. Ce filtrage permet d'être plus précis et d'éviter les faux positifs, 
 en ne se basant pas juste sur un mot ou une phrase mais des ensembles. 
 \item [List noire/blanche automatique] SpamAssassin garde automatiquement une liste blanches des expéditeurs des mails.
 Pour chaque nouveau mail le programme compare le mail précédent provenant de la même adfresse mail et adresse IP.
 Comme précédemment si une adresse email a envoyé un SPAM, ce nouveau mail verra son score abaissé. A l'inverse si c'était un bon mail son score 
 se verra baisser.
 \item [Liste noire/blanche manuelle] Il est tout à fait possible de définir ses propres listes, en autorisant ou 
 interdisant des mails de certaines adresses
 \item [Signalements] En utilisant des signatures établis à partir de mails signalés par les utilisateurs. Il y a notamment les projets DCC, Pyzor, et Razor2
 qui possèdent des bases de données de mails signalés comme SPAM. SpamAssassin va ainsi demander à ces bases si les mails qu'il reçoit 
 sont présents dans leurs donnée.
 \item [DNS blocklists] Ce sont des des bases de données contenant des adresses IP signalées comme expédiant du SPAM ou mal 
 configurée(Par exemple en étant un relais ouvert). Également sont pris en compte les IP de particuliers (considérant qu'il y a peu de chance
qu'un particulier envois directement des mails sans passer par son FAI. Ces signalements vont être pris en compte par SpamAsassin pour le score des mails.
Il intègre nativement quelque une de ces listes.
\item [Caractères et langues] on peut spécifier des caractères et langues comme SPAM.

C'est ces ensembles de règles qui fonctionnant conjointement permettent à SpamAssassin de garantir un haut 
niveau de fiabilité de détection des SPAM, un test pouvant ne pas fonctionner mais sera contrebalancé par les autres.
 \end{description} 


Spam Assassin effectue sur chaque mail qui lui est donné à traiter une série de test, qui vont ensuite donner lieu à un score, qui sera indiqué dans un entête si il est considéré comme SPAM. 
Ce résultat sera ensuite utilisé par d'autres programmes pour déterminer des actions à entreprendre.


\subsection{Le score} \label{score}
C'est la base du signalement des SPAM de SpamAsassin. Un mail après avoir subi des test différents se voit attribuer une note. Cette note permet
ensuite de définir des actions à effectuer. Quand un entête de mail est réécrit, SpamAssassin ajoute ses propres champs avec
notamment le score, mais également d'autres données (Exemple ~\ref{fig:ex_score}). L'utilisateur peut paramétrer ce 
score pour définir une marge de définitions des SPAMs(valeurs ''require``)

\begin{figure}[b]
 \includegraphics[width=\textwidth]{annexes/entete.png}
 \label{fig:ex_score}
\end{figure}

\pagebreak

\section{Articulation du programme}

\subsection{Configuration}

La configuration de Spamassassin se fait principalement à travers le fichier \emph{local.cf}, qui se trouve dans le 
repertoire '\emph{/etc/spamassassin}.
Les utilisateurs UNIX peuvent également avoir leurs propres configurations grâce au fichier \emph{user\_prefs}
qui se trouve dans le répertoire \emph{.spamassassin} de leurs home respectifs. 
Par défaut, un certain nombre d’options sont prédéfinies. Voici les principales \begin{itemize}
 \item 5 définit le score au delà duquel les mails sont considérés comme du spam ;
 \item fr en indique les langues que vous acceptez de recevoir (les autres auront un score plus élevé). Cette ligne n’est pas forcément prédéfinie ;
 \item fr pour avoir les rapports en Français ;
 \item *@domaine.org Cette ligne (à éditer) permet de ne pas considérer les mails du domaine comme du spam.
\end{itemize}

\subsubsection{Options de configuration}

Spam assassin possèdent plusieurs options dont certaine non activés par défaut. Il suffit de les ajouter aux fichiers 
de configuration.

\paragraph{Préférences utilisateurs}

\begin{description}
 \item [Options de score] \begin{description}
                           \item [required\_score n.nn (default:5)] Définit le score par défaut considérant un message comme SPAM. la valeur peut être un réel ou un entier
                           \item [score SYMBOLIC\_TEST\_NAME n.nn [ n.nn n.nn n.nn ]] Assigne les scores( voir ~\ref{score}) à un test donné.
                          \end{description}


\end{description}



\appendix





\end{document}          
